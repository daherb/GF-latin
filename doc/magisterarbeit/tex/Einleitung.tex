\section{Motivation}
\label{sec:motivation}
Im Bereich der Computerlinguistik haben sich im Laufe der Zeit zwei Lager gebildet, die jeweils ihren Ansatz zur Sprachverarbeitung vertreten. Der heute häufiger anzutreffende Ansatz ist der statistische Ansatz, denn nach aktuellem Stand kann man durch statistische Methoden in der Sprachverarbeitung mit relativ geringem Aufwand brauchbare bis gute Ergebnisse erzielen. Allerdings bedarf der statistische Ansatz möglichst guter Trainingsdaten, die nicht immer leicht zu beschaffen und zu bewerten sind. \par
Der zweite, ältere Ansatz, ist die regelbasierte Sprachverarbeitung. Er prägte die Anfänge der Computerlinguistik stark, wurde jedoch im Laufe der Zeit vom statistischen Ansatz verdrängt. Dies ist unter anderem auf die steigende Leistung heutiger Rechner zur Verarbeitung großer Datenmengen und vor allem auf die Fülle an Daten, die über das Internet verfügbar sind, zurückzuführen. Die Grundlagen regelbasierter Grammatiken sind in etwa so alt wie die Wissenschaft der Linguistik selbst und sollten auch weiterhin zum Wissen eines jeden Computerlinguisten gehören. \par
In dieser Arbeit werden an einem konkreten Beispiel die nötigen Schritte gezeigt, um eine computergestützte Grammatik für eine natürliche Sprache zu entwerfen. An der lateinischen Sprache wird exemplarisch gezeigt, wie ein Lexikon, ein Morphologiesystem und eine Syntax implementiert werden können, die sich in ein größeres, multilinguales Grammatiksystem einfügen lassen. Eine Sprache wie Latein, die zum einen für ihre Regelmäßigkeit aber auch für ihre linguistischen Besonderheiten bekannt ist, kann zu interessanten Erkenntnissen im Bereich der Grammatikentwicklung führen. \par
Ein weiterer Aspekt dieser Arbeit ist es, einen Beitrag zu einem größeren Projekt zu leisten. Denn die lateinische Grammatik, die im Rahmen dieser Arbeit entwickelt wurde, fließt direkt in das Grammatical Framework\footnote{\url{http://www.grammaticalframework.org/}}, das für diese Arbeit gewählten multilingualen Grammatik-, Parsing- und Übersetzungssystem ein.
\pagebreak
\section{Ziel der Arbeit}
\label{sec:ziel}
Ziel dieser Arbeit ist es, zum einen ein soweit funktionstüchtiges Grammatiksystem zu entwickeln, dass es in der Lage ist, grundlegende Sätze zu verarbeiten und durch die Integration in ein multilinguales Grammatiksystem in andere, moderne, Sprachen zu übersetzen. Zum anderen sollen aber auch die allgemeinen Schritte einer Grammatikentwicklung exemplarisch an der lateinischen Sprache dargelegt werden. \par
Der Schwerpunkt soll dabei zunächst auf der Nähe zu einer gewöhnlichen, im bayerischen Gymnasialunterricht verwendeten, lateinischen Schulgrammatik liegen. Dies spiegelt sich in der Abfolge der Schritte und in einigen Entscheidungen beim Entwurf der Grammatik wieder. So ist eine lateinische Grammatik grob in folgende Abschnitte unterteilt: Phonologie, Wortarten und Wortbildung, Morphologie und Syntax. Zwar werden die ersten drei Teile in dieser Arbeit nur kurz angeschnitten, die logische Folge der zwei verbleibenden Teile wird hier aber beibehalten. Zusätzlich wird ihnen der üblicherweise in Grammatiken nicht in dieser Form auffindbare Teil über die Lexikonentwicklung vorangestellt. \par
Die Grammatik soll zusätzlich im bestehenden System des Grammatical Frameworks verwendet werden können. Deshalb müssen möglichst große Teile der für eine Sprache vom Grammatical Framework geforderten Schnittstellen zur Verfügung gestellt werden. Diese Schnittstellen werden im Grammatical Framework in der Form einer so genannten abstrakten Syntax definiert, die im folgenden noch genauer beschrieben wird. So wird eine Menge von Modulen, Funktionen und Regeln definiert, die in einer Grammatik enthalten sein müssen. Das geplante Ziel ist es zu ermöglichen, dass alle von dieser Grammatik beschriebene Sätze in andere Sprachen, die von der Ressource Grammar Library des Grammatical Frameworks unterstützt werden, übersetzt werden können.
\pagebreak
\section{Aufbau der Arbeit}
\label{sec:aufbau}
Die Arbeit unterteilt sich in die drei Teile "`Grundlagen"', "`Grammatikentwicklung"' und ein Fazit. Die Grammatikentwicklung ist wiederum in die schon erwähnten Abschnitte über Lexikon, Morphologie und Syntax unterteilt. \par
Zu Beginn der Arbeit werden in Kapitel \ref{chap:grundlagen} die nötigen Grundlagen für die weiteren Teile der Arbeit erörtert. Die Grundlagen des Grammatical Frameworks werden in Abschnitt \ref{sec:gf} erklärt. Diese bestehen aus einer kurzen Beschreibung des Umfangs und der Funktionen dieses Programmpakets. Es folgt eine Einführung in den Formalismus, der bei der Entwicklung der Grammatiken für das Grammatical Framework verwendet wird. Dieser Abschnitt wird mit einigen Informationen zur Ressource Grammar Library, der multilingualen Grammatikbibliothek des Grammatical Frameworks, abgeschlossen. \par
Nach den technischen Grundlagen folgen einige Informationen zur lateinischen Sprache in Abschnitt \ref{sec:latein}. Zunächst wird eine sprachwissenschaftliche Einordnung dieser Sprache unter besonderer Hervorhebung einiger interessanter Merkmale versucht. Abschließend wird die Relevanz dieser als tot geltenden Sprache in der heutigen Zeit diskutiert. \par
Nach dieser allgemeinen Einführung werden im nächsten Kapitel, dem \mbox{Kapitel \ref{chap:grammatik}}, die nötigen Schritte, die umgesetzt wurden, um eine Lateingrammatik im Grammatical Framework zu entwickeln, beschrieben. Dieses Kapitel ist in die drei Abschnitte Lexikon, Morphologie und Syntax unterteilt, da diese drei getrennte Module in der Ressource Grammar Library bilden. Bei der Entwicklung der Grammatik zeigten sich allerdings oft Abhängigkeiten zwischen den drei Bestandteilen, so dass im Laufe der Zeit auch Änderungen in anderen Komponenten nötig waren. Im Abschnitt \ref{sec:lexikon} wird dargestellt, wie das Lexikon, das für eine Grammatik im Grammatical Framework nötig ist, erstellt wird. Darauf folgt in Abschnitt \ref{sec:morpho} die Beschreibung der lateinischen Wortflexion und wie sie im Grammatical Framework umgesetzt werden kann. Als letzter Teil dieses Kapitels wird in Abschnitt \ref{sec:syntax} erläutert, welche syntaktischen Regeln in der Grammatik nötig sind, um eine grundlegend funktionierende Grammatik zu erhalten, was zu den Hauptzielen dieser Arbeit gehört. \par
Abgerundet wird die Arbeit in Kapitel \ref{chap:fazit}, in dem ein Fazit der Arbeit gezogen und ein Ausblick auf mögliche Erweiterung und Verwendung gegeben wird. So wird gezeigt, welchen Sprachumfang die Grammatik bisher umfasst, welche Erweiterungen gewinnbringend sein können und auch in welchen Bereichen das Ergebnis dieser Arbeit Anwendung finden kann.
\section{Konventionen}
Zum Schluss dieser Einleitung sollen noch kurz die in dieser Arbeit verwendeten typografischen Konventionen erläutert werden. Grammatik-Quelltexte des Grammatical Frameworks, so wie Teile daraus, Befehle auf der Be\-triebs\-sys\-tem-Ein\-ga\-be\-auf\-for\-de\-rung und Befehle in einer interaktiven Sitzung des Grammatical Framework-Systems werden in einer \texttt{schreibmaschinenähnlichen Schrift} gesetzt. Befehle, die in der Be\-triebs\-sys\-tem-Ein\-ga\-be\-auf\-for\-de\-rung einzugeben sind, sind an einer Ein\-ga\-be\-auf\-for\-de\-rung in Form von \texttt{\$} vor dem Befehl und Befehle im Grammatical Framework-System an einem \texttt{>} zu erkennen. So ist \texttt{cat S;} ein Ausschnitt aus einem Quelltext, \mbox{\texttt{\$ gf}} ein Befehl in der Betriebssystem-Ein\-ga\-be\-auf\-for\-de\-rung und \mbox{\texttt{> generate\_random}} ein Befehl in einer interaktiven Grammatical Framework-Sitzung. \par
Alle Dateinamen und Pfade werden \textbf{fett} dargestellt. Ist bei einer Datei kein Pfad angegeben, so befindet sie sich im Verzeichnis, in dem die Lateingrammatik enthalten ist. Dieses Verzeichnis befindet sich entweder auf der beiliegenden CD oder im offiziellen Quelltext des Grammatical Frameworks im Unterverzeichnis \textbf{lib/src/latin}. Ist dagegen ein Ordnerpfad angegeben, so ist dieser relativ zu einem Verzeichnis, in dem sich der Quelltext des Grammatical Frameworks befindet. In Quelltext-Listings werden zur einfacheren Lesbarkeit auch Schlüsselwörter zusätzlich \texttt{\textbf{fett}} gedruckt.\par
Lateinische Wörter werden zur leichteren Erkennbarkeit \textit{kursiv} gesetzt. Zusätzlich sind in Quelltext-Listings die Inhalte von Zeichenketten ebenfalls \texttt{\textit{kursiv}}. \par
Endet eine Zeile in einem Listing mit einem umgekehrten Schrägstrich ("`\textbackslash"') so bedeutet dies, dass lediglich aus Platzgründen ein Zeilenumbruch eingefügt werden musste und die nächste Zeile eine Fortführung der Zeile an dieser Stelle darstellt.